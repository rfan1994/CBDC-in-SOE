\documentclass[12pt]{article}
 
\usepackage[margin=1in]{geometry} 
\usepackage{amsmath,amsthm,amssymb}
\usepackage{graphicx}
\usepackage{bbold}
\usepackage{multirow}

\newcommand{\N}{\mathbb{N}}
\newcommand{\Z}{\mathbb{Z}}
\newcommand{\E}{\mathbb{E}}
\newcommand{\Prob}{\text{Prob}}


\begin{document}
\title{\large \textbf{The TANK Model with CBDC, Bank and Dollarization}}
\date{}
\maketitle
\section{Model without Dollarization}
This model is an extension of the TANK model in a small open economy, with CBDC and a monopolistic competitive bank sector. Deposit, cash and CBDC provide liquidity service and decrease the transaction cost. Banks take deposits from unconstrained households and invest in intermediate firms; constrained households do not have access to bank services and can only hold cash or CBDC. 

\subsection{Households}
The consumption bundle is defined as follows:
\begin{align*}
c_t = [\gamma^{\frac{1}{\eta}}c_{Ht}^{\frac{\eta-1}{\eta}}+(1-\gamma)^{\frac{1}{\eta}}c_{Ft}^{\frac{\eta-1}{\eta}}]^{\frac{\eta}{\eta-1}}, 
\end{align*}
where $c_{Ht}$ and $c_{Ft}$ denote consumption of domestic and foreign final good respectively. $\gamma$ is the home bias. The representative household decides how to allocate her consumption expenditure between domestic and foreign goods.

By solving the static optimization problem, the optimal consumption of domestic and foreign final goods can be solved as 
\begin{align*}
c_{Ht} &= \gamma(\frac{P_{Ht}}{p_t})^{-\eta}, \\
c_{Ft} &= (1-\gamma)(\frac{P_{Ft}}{p_t})^{-\eta}, 
\end{align*}
where $p_t$ is the domestic CPI, i.e., the price of one unit of consumption: 
\begin{align*}
p_t = [\gamma P_{Ht}^{1-\eta}+(1-\gamma)P_{Ft}^{1-\eta}]^{\frac{1}{1-\eta}}.
\end{align*}
In a small open economy, the price of the foreign good coincides with the foreign CPI $p_t^*$, adjusted by the nominal exchange rate $e_t$:
\begin{align*}
P_{Ft} = e_tp_t^*
\end{align*}
The real exchange rate is defined as the ratio of the foreign and domestic price levels, where the foreign price level is converted into domestic currency units via the nominal exchange rate: 
\begin{align*}
s_t = e_t \frac{p_t^*}{p_t}
\end{align*}
Define $p_{Ht} = P_{Ht}/p_t$ and  $p_{Ft} = P_{Ft}/p_t$ as the price of domestic and foreign goods in terms of the domestic CPI, then the real exchange rate can be written as: 
\begin{align*}
s_t = p_{Ft}, 
\end{align*}
and the terms of trade, i.e., the ratio between the price imports and the price exports can be written as 
\begin{align*}
tot_t = \frac{P_{Ft}}{P_{Ht}} = \frac{s_t}{p_{Ht}}
\end{align*}

\subsubsection*{Unconstrained HH ($1-\lambda$)} 
Unconstrained households with measure $1-\lambda$ have access to the bank service. Each period, they choose their consumption $c_{1t}$ and labor supply $h_{1t}$ and receive labor income $\omega_t$. Unconstrained households pay consumption tax at the rate $\tau_c$. They also receive a lump-sum transfer from the government $t_{1t}$ and profit of intermediate firms and banks $\Gamma_{1t}$.

Unconstrained households face liquidity constraints. I follow Schmitt-Groh{\'e} and Uribe (2007) and assume that the transaction cost of consumption $s_{1t}$ is determined by the Transaction Cost Function:
\begin{align*}
s_{1t} = z_tA_1\frac{c_{1t}}{l_{1t}}+B_1\frac{l_{1t}}{c_{1t}}-2\sqrt{A_1B_1},
\end{align*}
where the ratio between consumption and liquidity $c_{1t}/l_{1t}$ represents the money velocity, the transaction cost is increasing in the velocity as long as the following condition is satisfied:  
\begin{align*}
\frac{c_{1t}}{l_{1t}}>\sqrt{\frac{B_1}{z_tA_1}}.
\end{align*} 

Unconstrained households get liquidity services from commercial banks. There is a continuum measure of monopolistic competitive banks index by $j \in [0,1]$. Bank deposits are substitutes with the elasticity of substitution $\epsilon_b > 1$. The total liquidity for unconstrained households can be written as: 
\begin{align}
\label{l1t}
l_{1t} = \int_0^1({d_{1t}(j)}^{\frac{\epsilon_b-1}{\epsilon_b}}dj)^{\frac{\epsilon_b}{\epsilon_b-1}} 
\end{align}

The return to deposit $r_t^d(j)$ is determined by the bank sector optimization problem. Unconstrained households also hold one-period government bond $b_{1Ht}$ and foreign bond $b_{1Ft}$, with $r_t$ and $r_t^*$ being the nominal interest rate of bonds. 

Domestic households pay a quadratic adjustment cost when they change their financial position (foreign bond and foreign currency) with the rest of the world: this assumption ensures the existence of a determinate steady state and a stationary solution. (Schmitt-Groh{\'e} and Uribe, 2003)

The unconstrained households' problem can be defined as (all variables are in real terms):
\begin{align*}
\max_{c_{1t}, h_{1t},s_{1t},l_{1t},d_{1t}(j),b_{1Ht},b_{1Ft}} E_0 \sum_0^{\infty}\beta^t (\frac{c_{1t}^{1-\sigma}}{1-\sigma}-\chi\frac{h_{1t}^{1+\phi}}{1+\phi}),
\end{align*}
subject to the budget and liquidity constraints:
\begin{align*} 
&(1+s_{1t}+\tau_c)c_{1t}+\int_0^1(d_{1t}(j)-\frac{r_{t-1}^d(j)}{\pi_t}d_{1t-1}(j))dj +(b_{1Ht}-\frac{r_{t-1}}{\pi_t}b_{1Ht-1})+s_t(b_{1Ft}-r^*_{t-1}b_{1Ft-1}) \\
&\leq w_th_{1t}+t_{1t}+\Gamma_{1t}-\frac{\kappa_B}{2}s_t((1-\lambda)b_{1Ft}-\bar{b}_F)^2 \quad (\lambda_{1t}) \\
&l_{1t} = \int_0^1({d_{1t}(j)}^{\frac{\epsilon_b-1}{\epsilon_b}}dj)^{\frac{\epsilon_b}{\epsilon_b-1}} \quad (\tau_{1t}\lambda_{1t}) \\
&s_{1t} = z_tA_1\frac{c_{1t}}{l_{1t}}+B_1\frac{l_{1t}}{c_{1t}}-2\sqrt{A_1B_1}  \quad (\mu_{1t}\lambda_{1t})
\end{align*}
Taking first order conditions with respect to $c_{1t}, h_{1t}, s_{1t}, l_{1t}, d_{1t}(j), b_{1Ht}, b_{1Ft}$: 
\begin{align}
\label{c1}
c_{1t}: \quad &c_{1t}^{-\sigma}-\lambda_{1t}(1+s_{1t}+\tau_c)+\tau_{1t}\lambda_{1t}(z_tA_1\frac{1}{l_{1t}}-B_1\frac{l_{1t}}{c_{1t}^2}) = 0, \\
\label{h1}
h_{1t}: \quad &-\chi h_{1t}^{\phi}+\lambda_{1t}\omega_t  = 0, \\
\label{s1}
s_{1t}: \quad &-\lambda_{1t}c_{1t}-\tau_{1t}\lambda_{1t} = 0, \\
\label{l1}
l_{1t}: \quad &\tau_{1t}\lambda_{1t}(-z_tA_1\frac{c_{1t}}{l_{1t}^2}+B_1\frac{1}{c_{1t}})-\mu_{1t}\lambda_{1t} = 0, \\
\label{d1j}
d_{1t}(j): \quad &-\lambda_{1t}+\beta E_t\lambda_{1t+1}\frac{r_{t}^d(j)}{\pi_{t+1}}+\mu_{1t}\lambda_{1t}(\frac{l_{1t}}{d_{1t}(j)})^{\frac{1}{\epsilon_b}} = 0, \\
\label{b1H}
b_{1Ht}: \quad &-\lambda_{1t}+\beta E_t\lambda_{1t+1}\frac{r_t}{\pi_{t+1}} = 0, \\
\label{b1F}
b_{1Ft}: \quad &-\lambda_{1t}+\beta E_t\lambda_{1t+1}\frac{s_{t+1}}{s_t}t_t^*-\lambda_{1t}\kappa_B(1-\lambda)((1-\lambda)b_{1Ft}-\bar{b}_F) = 0,
\end{align}
Define the transaction wedge as:
\begin{align}
\label{tau1c}
\tau_{1t}^c = z_tA_1\frac{c_{1t}}{l_{1t}}-B_1\frac{l_{1t}}{c_{1t}}.
\end{align}
Plug (\ref{tau1c}) into first order condition (\ref{c1}) and (\ref{h1}), the intertemporal and intratemporal euler equation can be written as: 
\begin{align*}
c_{1t}^{-\sigma} &= \lambda_{1t}(1+s_{1t}+\tau_c+\tau_{1t}^c) \\
\lambda_{1t}(1-\frac{c_{1t}}{l_{1t}}\tau_{1t}^c(\frac{l_{1t}}{d_{1t}(j)})^{\frac{1}{\epsilon_b}}) &= \beta \E_t\lambda_{1t+1}\frac{r_{t}^d(j)}{\pi_{t+1}} \\
\chi h_{1t}^{\phi} &= \lambda_{1t}\omega_t 
\end{align*}
Combine equation (\ref{l1}), (\ref{d1j}) and (\ref{b1H}), then plug in (\ref{tau1c}), we can get: 
\begin{align*}
\beta E_t\lambda_{1t+1}\frac{r_{Ht}^d(j)}{\pi_{t+1}} = \beta E_t\lambda_{1t+1}\frac{r_t}{\pi_{t+1}}(1-\frac{c_{1t}}{l_{1t}}\tau_{1t}^c(\frac{d_{1Ht}}{d_{1Ht}(j)})^{\frac{1}{\epsilon_b}}). 
\end{align*}
Then plug the equation above to equation (\ref{l1t}), the demand function for $d_{1t}(j)$ can be derived as:
\begin{align}
d_{1t}(j) = \Biggl(\frac{(r_t-r_t^d(j))^{-\epsilon_b}}{\big(\int_0^1(r_t-r_t^d(j))^{1-\epsilon_b}dj\big)^{-\frac{\epsilon_b}{1-\epsilon_b}}}\Biggl)d_{1t}
\end{align}

\subsubsection*{Constrained HH ($\lambda$)}
The constrained households with measure $\lambda$ don't have access to bank services, so they don't have access to bank deposits or government bonds. Each period, they choose their consumption $c_{2t}$ and labor supply $h_{2t}$ and receive labor income $\omega_t$. Unconstrained households pay consumption tax at rate $\tau_c$, but only when the payment is made by CBDC. They also receive a lump-sum transfer from the government $t_{2t}$. 

Constrained households face the same liquidity constraints as unconstrained households. They can only get liquidity service by holding cash ($m_{2t}$) or CBDC ($CBDC_{2t}$). Cash users are facing an additional cost $\delta_m$ since cash is at risk of being stolen and subject to an adjustment cost. CBDC is a safer and more convenient liquid asset, so it's free of risk and adjustment costs. Cash and CBDC are substitutes with the elasticity of substitution $\epsilon_m>1$. The total liquidity for constrained households can be written as: 
\begin{align*}
l_{2t} = ((m_{2t})^{\frac{\epsilon_m-1}{\epsilon_m}}+(CBDC_{2t})^{\frac{\epsilon_m-1}{\epsilon_m}})^{\frac{\epsilon_m}{\epsilon_m-1}}
\end{align*}

The constrained households' problem can be defined as (all variables are in real terms):
\begin{align*}
\max_{c_{2t}, h_{2t},l_{2t},s_{2t},m_{2t},CBDC_{2t}} E_0 \sum_0^{\infty}\beta^t (\frac{c_{2t}^{1-\sigma}}{1-\sigma}-\chi\frac{h_{2t}^{1+\phi}}{1+\phi})
\end{align*}
subject to the budget and liquidity constraints:
\begin{align*} 
\text{s.t.} \quad & (1+s_{2t}+\tau_c\frac{CBDC_{2t}}{l_{2t}})c_{2t}+(m_{2t}-\frac{1-\delta_m}{\pi_t}m_{2t-1})+(CBDC_{2t}-\frac{r_{t-1}^{CBDC}}{\pi_t}CBDC_{2t-1}) \\
&\leq w_th_{2t}+t_{2t}\\
&l_{2t} = ((m_{2t})^{\frac{\epsilon_m-1}{\epsilon_m}}+(CBDC_{2t})^{\frac{\epsilon_m-1}{\epsilon_m}})^{\frac{\epsilon_m}{\epsilon_m-1}}  \quad (\tau_{2t}\lambda_{2t})\\
& s_{2t} = z_tA_2\frac{c_{2t}}{l_{2t}}+B_2\frac{l_{2t}}{c_{2t}}-2\sqrt{A_2B_2} \quad (\mu_{2t}\lambda_{2t})
\end{align*}
Taking first order conditions with respect to $c_{2t}, h_{2t}, s_{2t}, l_{2t}, m_{2t}, CBDC_{2t}$: 
\begin{align}
\label{c2}
c_{2t}: \quad &c_{2t}^{-\sigma}-\lambda_{2t}(1+s_{2t}+\tau_c\frac{CBDC_{2t}}{l_{2t}})+\tau_{2t}\lambda_{2t}(z_tA_2\frac{1}{l_{2t}}-B_2\frac{l_{2t}}{c_{2t}^2}) = 0, \\
\label{h2}
h_{2t}: \quad &-\chi h_{2t}^{\phi}+\lambda_{2t}\omega_t  = 0, \\
\label{s2}
s_{2t}: \quad &-\lambda_{2t}c_{2t}-\tau_{2t}\lambda_{2t} = 0, \\
\label{l2}
l_{2t}: \quad &\lambda_{2t}\tau_c\frac{CBDC_{2t}}{l_{2t}^2}c_{2t}+\tau_{2t}\lambda_{2t}(-z_tA_2\frac{c_{2t}}{l_{2t}^2}+B_2\frac{1}{c_{2t}})-\mu_{2t}\lambda_{2t} = 0, \\
\label{m2}
m_{2t}: \quad &-\lambda_{2t}+\beta E_t\lambda_{2t+1}\frac{1-\delta_m}{\pi_{t+1}}+\mu_{2t} \lambda_{2t}(\frac{l_{2t}}{m_{2t}})^{\frac{1}{\epsilon_m}}= 0, \\
\label{CBDC2}
CBDC_{2t}: \quad &-\lambda_{2t}(1+\tau_c\frac{c_{2t}}{l_{2t}})+\beta E_t\lambda_{2t+1}\frac{r_{t}^{CBDC}}{\pi_{t+1}}+\mu_{2t} \lambda_{2t}(\frac{l_{2t}}{CBDC_{2t}})^{\frac{1}{\epsilon_m}}= 0.
\end{align}
Similar to the unconstrained households, we can define the transaction wedge as:
\begin{align}
\label{tau2c}
\tau_{2t}^c = z_tA_2\frac{c_{2t}}{l_{2t}}-B_2\frac{l_{2t}}{c_{2t}}.
\end{align}
Plug (\ref{tau2c}) into first order condition (\ref{c2}) and  (\ref{h2}), combine with equation (\ref{m2}), the intertemporal and intratemporal euler equation can be written as: 
\begin{align*}
c_{2t}^{-\sigma} &= \lambda_{2t}(1+s_{2t}+\tau_{2t}^c+\tau_c\frac{CBDC_{2t}}{l_{2t}}) \\
\lambda_{2t}(1-\frac{c_{2t}}{l_{2t}}(\tau_{2t}^c+\tau_c\frac{CBDC_{2t}}{l_{2t}})(\frac{l_{2t}}{m_{2t}})^{\frac{1}{\epsilon_m}}) &= \beta E_t \lambda_{2t+1} \frac{1-\delta_m}{\pi_{t+1}}, \\
\chi h_{1t}^{\phi} &= \lambda_{1t}\omega_t 
\end{align*}
The no-arbitrage condition for cash and CBDC can be derived from equations (\ref{m2}) and (\ref{CBDC2}).

\subsection{Bank}
There is a continuum measure of monopolistic competitive banks $j \in [0,1]$. They take deposit from unconstrained households $d_{t}(j)$ and invest in intermediate firms $i_t(j)$ and accumulate capital stock $k_t(j)$. The deposit demand functions are solved from the household problem above. 
\begin{align*}
d_{t}(j) = \Biggl(\frac{(r_t-r_t^d(j))^{-\epsilon_b}}{\big(\int_0^1(r_t-r_t^d(j))^{1-\epsilon_b}dj\big)^{-\frac{\epsilon_b}{1-\epsilon_b}}}\Biggl)d_{t} 
\end{align*}

Banks are owned by unconstrained households, so bankers maximize its expected pre-dividend profits: 
\begin{align*}
 &\max_{r_t^d(j),i_t(j),k_t(j)}E_0 \sum_0^{\infty}\beta^t\frac{\lambda_{1t}}{\lambda_{10}}(r_t^kk_{t-1}(j)-i_t(j)+(d_{t}(j)-\frac{r_{t-1}^d(j)}{\pi_t}d_{t-1}(j)),
 \end{align*}
subject to the law of motion of capital and balance sheet constraint: 
 \begin{align*}
  \quad & k_t(j) = (1-\delta)k_{t-1}(j)+(1-\frac{\kappa_I}{2}(\frac{i_t(j)}{i_{t-1}(j)}-1))i_t(j), \\
& k_t(j) = d_{t}(j).
\end{align*}

The equilibrium is symmetric, then the first-order conditions give the supply for new investment 
\begin{align*}
\lambda_{1t} = q_t\lambda_{1t}(1-\frac{\kappa_I}{2}(\frac{i_t}{i_{t-1}}-1)^2-\kappa_I(\frac{i_t}{i_{t-1}}-1)\frac{i_t}{i_{t-1}})+ \beta\E_tq_{t+1}\lambda_{1t+1}\kappa_I(\frac{i_{t+1}}{i_t}-1)^2(\frac{i_{t+1}}{i_t})^2,
\end{align*}
and optimal deposit rate chosen by the bank:  
\begin{align*}
{r_{t}^d}^* &= \frac{\epsilon_b}{\epsilon_b-1}\frac{\beta E_t \lambda_{1t+1}(r_{t+1}^k+(1-\delta)q_{t+1})-(q_t-1)\lambda_{1t}}{\beta E_t (\lambda_{1t+1}/\pi_{t+1})}-\frac{1}{\epsilon_b-1}r_t.
\end{align*}
When $\epsilon_b \to \infty$, then banks are perfectly competitive, then the deposit rate chosen by the bank becomes the net return to capital: 
\begin{align*}
{r_{t}^d}^* &= \frac{\beta E_t \lambda_{1t+1}(r_{t+1}^k+(1-\delta)q_{t+1})-(q_t-1)\lambda_{1t}}{\beta E_t (\lambda_{1t+1}/\pi_{t+1})}.
\end{align*}

\subsection{Firm}
The firm problem is standard. The representative final-good firm uses the following CES aggregator to produce the domestic final good, and the intermediate firms produce differentiated domestic input. 

\subsubsection*{Final}
The representative final-good firm uses the following CES aggregator to produce the domestic final good $y_{Ht}$:
\begin{align*}
y_{Ht} = (\int_0^1 y_{Ht}(i)^{\frac{\epsilon}{\epsilon-1}})^{\frac{\epsilon-1}{\epsilon}}
\end{align*}
where $y_{Ht}(i)$ is an intermediate input produced by the intermediate firm $i$, whose price is $P_{Ht}(i)$. The final good firm maximizes the profit: 
\begin{align*}
\max_{y_{Ht},y_{Ht}(i)}\quad P_{Ht}(\int_0^1 y_{Ht}(i)^{\frac{\epsilon}{\epsilon-1}})^{\frac{\epsilon-1}{\epsilon}}- \int_0^1 P_{Ht}(i)y_{Ht}(i)
\end{align*}
Then the first order conditions give the demand for intermediate goods: 
\begin{align*}
y_{Ht}(i) = y_{Ht}\Big(\frac{P_{Ht}(i)}{P_{Ht}}\Big)^{-\epsilon}
\end{align*}

\subsubsection*{Intermediate}
There is a continuum of firms indexed by $i$ producing a differentiated domestic input using the following Cobb-Douglas function:
\begin{align*}
y_{Ht}(i) = a_t(k_{t-1}(i))^{\alpha}(h_t(i))^{1-\alpha}
\end{align*}
where $a_t$ is the total factor productivity, which follows an autoregressive process: 
\begin{align*}
\log(a_t) = (1-\rho_a)\log(\bar{a})+\rho_a\log(a_{t-1})+\nu_t^a, \quad  \nu_t^a \sim N(0,\sigma_a^2)
\end{align*}
Firms operate in monopolistic competition, so they set the price of their own good subject to the demand of final good firms. In addition, firms pay quadratic adjustment cost $AC_t(i)$ in nominal terms as in Rotemberg (1982), whenever they adjust prices with respect to the benchmark inflation rate $\bar{\pi}$:
\begin{align*}
AC_t(i) = \frac{\kappa_P}{2}\Big(\frac{P_{Ht}(i)}{P_{Ht-1}(i)}-\bar{\pi}\Big)^2P_{Ht}y_{Ht}
\end{align*}
The intermediate firm is owned by unconstrained households and maximizes the profit: 
\begin{align*}
\max_{P_{Ht}(i),y_{Ht}(i),k_{t-1}(i), h_t(i)} &E_0 \sum_0^{\infty}\beta^t\frac{\lambda_{1t}}{\lambda_{10}}(\frac{P_{Ht}(i)}{p_t}y_{Ht}(i)-w_th_t(i)-r_t^kk_{t-1}(i)-\frac{AC_t(i)}{p_t}),
\end{align*}
subject to the production and demand function: 
\begin{align*}
y_{Ht}(i) &= a_t(k_{t-1}(i))^{\alpha}(h_t(i))^{1-\alpha}, \\
y_{Ht}(i) &= y_{Ht}\Big(\frac{P_{Ht}(i)}{P_{Ht}}\Big)^{-\epsilon}.
\end{align*}

The equilibrium is symmetric, the equilibrium interest rate and wage can be solved as: 
\begin{align*}
r_t^k &= \alpha mc_t \frac{y_{Ht}}{k_{t-1}}, \\
\omega_t &= (1-\alpha) mc_t \frac{y_{Ht}}{h_t}.
\end{align*}
The marginal cost $mc_t$ is defined by reranging the price condition: 
\begin{align*}
\frac{\epsilon}{\kappa_P}(\frac{mc_t}{p_{Ht}}-\frac{\epsilon-1}{\epsilon}) = (\pi_{Ht}-\bar{\pi})\pi_t-\beta E_t \frac{\lambda_{1t+1}}{\lambda_{1t}} (\pi_{Ht+1}-\bar{\pi})\pi_{Ht+1}\frac{y_{Ht+1}}{y_{Ht}}
\end{align*}
\subsection{Policy}
The government finances public expenditure $g_t$ and transfers $\tau_{1t}$ and $\tau_{2t}$ by raising consumption tax and public debt:
\begin{align*}
g_t + (1-\lambda)t_{1t}+\lambda t_{2t} &= (1-\lambda)(b_{1t}-\frac{r_{t-1}}{\pi_t}b_{1t-1})+\lambda(CBDC_{2t}-\frac{1}{\pi_t} CBDC_{2t-1} ) \\
& + (1-\lambda)\tau_c c_{1t} +\lambda\tau_c\frac{CBDC_{2t}}{l_{2t}}c_{2t}
\end{align*}
The central bank sets interest rate following the Taylor rule: 
\begin{align*}
\frac{r_t}{\bar{r}} = (\frac{r_{t+1}}{\bar{r}})^{\rho_r}((\frac{\pi_t}{\bar{\pi}} )^{\phi_{\pi}} (\frac{p_{Ht}y_{Ht}}{\bar{p_Hy_H}})^{\phi_y} (\frac{\Delta e_t}{\bar{\Delta e}})^{\phi_e})^{1-\rho_r}\exp(\nu_t^m)
\end{align*}

\subsection{Foreign Sector}
Given that the domestic economy is sufficiently small relatively to the foreign economy, the foreign sector is treated as endogenously given. The foreign output $y_t^*$ and interest rate $r_t^*$ follow the autoregressive processes: 
\begin{align*}
\log(y_t^*) &= \rho_y\log(y_{t-1}^*)+\nu_t^y \\
r_t^*&= (1-\rho_r)\frac{1}{\beta}+\rho_rr_{t-1}^*+\nu_t^r 
\end{align*}
The foreign CPI is assumed to be constant over time:
\begin{align*}
\pi^* = \frac{p_t^*}{p_{t-1}^*} = 1
\end{align*}
Given the home bias $\gamma^*$ of the foreign countr, the demand for domestic good of foreign sector is 
\begin{align*}
\gamma^*\Big(\frac{p_{Ht}}{s_t}\Big)^{-\eta}y_t^*
\end{align*}
\section{Model with Dollarization}
In this section, I extend the model so that deposits and cash can be indexed by domestic or foreign currency. CBDC can only be indexed by domestic currency. 
\subsection{Households}
\subsubsection*{Unconstrained HH ($1-\lambda$)} 
The unconstrained households now can choose the deposit to be indexed by domestic $d_{1Ht}(j)$ or foreign currency $d_{1Ft}(j)$; the deposit indexed by domestic and foreign currency are perfect substitutes. The total liquidity for unconstrained households can be written as: 
\begin{align*}
l_{1t} = \int_0^1({d_{1Ht}(j)}^{\frac{\epsilon_b-1}{\epsilon_b}}dj)^{\frac{\epsilon_b}{\epsilon_b-1}}+s_t\int_0^1({d_{1Ft}(j)}^{\frac{\epsilon_b-1}{\epsilon_b}}dj)^{\frac{\epsilon_b}{\epsilon_b-1}} 
\end{align*}
The return to deposit indexed by domestic and foreign currency $r_{Ht}^d(j)$ and $r_{Ft}^d(j)$ are determined by the bank sector problem. 

Domestic households pay a quadratic adjustment cost when they change their financial position (foreign bond and foreign currency) with the rest of the world: this assumption ensures the existence of a determinate steady state and a stationary solution. (Schmitt-Groh{\'e} and Uribe, 2003)

The unconstrained households' problem can be defined as (all variables are in real terms):
\begin{align*}
\max_{c_{1t}, h_{1t},s_{1t},l_{1t},d_{1Ht}(j),d_{1Ft}(j),b_{1Ht},b_{1Ft}} E_0 \sum_0^{\infty}\beta^t (\frac{c_{1t}^{1-\sigma}}{1-\sigma}-\chi\frac{h_{1t}^{1+\phi}}{1+\phi}),
\end{align*}
subject to the budget and liquidity constraints:
\begin{align*} 
&(1+s_{1t}+\tau_c)c_{1t}+\int_0^1(d_{1Ht}(j)-\frac{r_{Ht-1}^d(j)}{\pi_t}d_{1Ht-1}(j))dj+s_t\int_0^1(d_{1Ft}(j)-r_{Ft-1}^d(j)d_{1Ft-1}(j))dj \\
&+(b_{1Ht}-\frac{r_{t-1}}{\pi_t}b_{1Ht-1})+s_t(b_{1Ft}-r^*_{t-1}b_{1Ft-1}) \\
&\leq w_th_{1t}+t_{1t}+\Gamma_{1t}-\frac{\kappa_D}{2}s_t((1-\lambda)\int_0^1d_{1Ft}(j)dj-\bar{d}_F)^2-\frac{\kappa_B}{2}s_t((1-\lambda)b_{1Ft}-\bar{b}_F)^2 \quad (\lambda_{1t}) \\
&l_{1t} = \int_0^1({d_{1Ht}(j)}^{\frac{\epsilon_b-1}{\epsilon_b}}dj)^{\frac{\epsilon_b}{\epsilon_b-1}}+s_t\int_0^1({d_{1Ft}(j)}^{\frac{\epsilon_b-1}{\epsilon_b}}dj)^{\frac{\epsilon_b}{\epsilon_b-1}}  \quad (\tau_{1t}\lambda_{1t})\\
&s_{1t} = z_tA_1\frac{c_{1t}}{l_{1t}}+B_1\frac{l_{1t}}{c_{1t}}-2\sqrt{A_1B_1}  \quad (\mu_{1t}\lambda_{1t})
\end{align*}
Taking first order conditions with respect to $c_{1t}, h_{1t}, s_{1t}, l_{1t}, d_{1Ht}(j), d_{1Ft}(j), b_{1Ht}, b_{1Ft}$: 
\begin{align*}
c_{1t}: \quad &c_{1t}^{-\sigma}+\lambda_{1t}(1+s_{1t}+\tau_c)-\tau_{1t}\lambda_{1t}(z_tA_1\frac{1}{l_{1t}}-B_1\frac{l_{1t}}{c_{1t}^2}) = 0, \\
h_{1t}: \quad &-\chi h_{1t}^{\phi}+\lambda_{1t}\omega_t  = 0, \\
s_{1t}: \quad &-\lambda_{1t}c_{1t}-\tau_{1t}\lambda_{1t} = 0, \\
l_{1t}: \quad &\tau_{1t}\lambda_{1t}(-z_tA_1\frac{c_{1t}}{l_{1t}^2}+B_1\frac{1}{c_{1t}})-\mu_{1t}\lambda_{1t} = 0, \\
d_{1Ht}(j): \quad &-\lambda_{1t}+\beta E_t\lambda_{1t+1}\frac{r_{Ht}^d(j)}{\pi_{t+1}}+\mu_{1t}\lambda_{1t}(\frac{d_{1Ht}}{d_{1Ht}(j)})^{\frac{1}{\epsilon_b}} = 0, \\
\begin{split}
d_{1Ft}(j): \quad &-\lambda_{1t}+\beta E_t\lambda_{1t+1}\frac{s_{t+1}}{s_t}r_{Ft}^d(j)+\mu_{1t}\lambda_{1t}(\frac{d_{1Ft}}{d_{1Ft}(j)})^{\frac{1}{\epsilon_b}}  \\
&-\lambda_{1t}\kappa_D(1-\lambda)((1-\lambda)\int_0^1d_{1Ft}(j)dj-\bar{d}_F) = 0, 
\end{split} \\
b_{1Ht}: \quad &-\lambda_{1t}+\beta E_t\lambda_{1t+1}\frac{r_t}{\pi_{t+1}} = 0, \\
b_{1Ft}: \quad &-\lambda_{1t}+\beta E_t\lambda_{1t+1}\frac{s_{t+1}}{s_t}t_t^*-\lambda_{1t}\kappa_B(1-\lambda)((1-\lambda)b_{1Ft}-\bar{b}_F) = 0,
\end{align*}
with 
\begin{align*}
d_{1Ht} &= \int_0^1({d_{1Ht}(j)}^{\frac{\epsilon_b-1}{\epsilon_b}}dj)^{\frac{\epsilon_b}{\epsilon_b-1}}, \\
d_{1Ft} &= \int_0^1({d_{1Ft}(j)}^{\frac{\epsilon_b-1}{\epsilon_b}}dj)^{\frac{\epsilon_b}{\epsilon_b-1}}.
\end{align*}
Similar to the case without dollarization, the demand function for $d_{1Ht}(j)$  and $d_{1Ft}(j)$ can be derived as:
\begin{align*}
d_{1Ht}(j) &= \Biggl(\frac{(r_t-r_{Ht}^d(j))^{-\epsilon_b}}{\big(\int_0^1(r_t-r_{Ht}^d(j))^{1-\epsilon_b}dj\big)^{-\frac{\epsilon_b}{1-\epsilon_b}}}\Biggl)d_{1Ht} \\
d_{1Ft}(j) &= \Biggl(\frac{(\frac{r_t^*}{1+\tau_{Bt}}-\frac{r_{Ft}^d(j)}{1+\tau_{Dt}})^{-\epsilon_b}}{\big(\int_0^1(\frac{r_t^*}{1+\tau_{Bt}}-\frac{r_{Ft}^d(j)}{1+\tau_{Dt}})^{1-\epsilon_b}dj\big)^{-\frac{\epsilon_b}{1-\epsilon_b}}}\Biggl)d_{1Ht}
\end{align*}
where $\tau_{Bt}$ and $\tau_{Dt}$ are wedges due to the quadratic adjustment costs: 
\begin{align*}
\tau_{Bt} &= \kappa_B(1-\lambda)((1-\lambda)b_{1Ft}-\bar{b}_F) \\
\tau_{Dt} &= \kappa_D(1-\lambda)((1-\lambda)d_{1Ft}-\bar{d}_F)
\end{align*}

\subsubsection*{Constrained HH  ($\lambda$)}
The constrained households now can choose to hold domestic $m_{2Ht}$ or foreign cash $m_{2Ft}$. CBDC is only indexed in domestic currency. 
Domestic cash users are facing an additional cost $\delta_m$, since cash is under the risk of being stolen and subject to an adjustment cost. Domestic and foreign currency are perfect substitutes. Cash and CBDC are substitutes with elasticity of substitution $\epsilon_m>1$ . The total liquidity for constrained households can be written as: 
\begin{align*}
l_{2t} = ((m_{2Ht})^{\frac{\epsilon_m-1}{\epsilon_m}}+(s_t m_{2Ft})^{\frac{\epsilon_m-1}{\epsilon_m}}+(CBDC_{2t})^{\frac{\epsilon_m-1}{\epsilon_m}})^{\frac{\epsilon_m}{\epsilon_m-1}}
\end{align*}

The constrained households' problem can be defined as (all variables are in real terms):
\begin{align*}
\max_{c_{2t}, h_{2t},l_{2t},s_{2t},m_{2Ht},m_{2Ft}} E_0 \sum_0^{\infty}\beta^t (\frac{c_{2t}^{1-\sigma}}{1-\sigma}-\chi\frac{h_{2t}^{1+\phi}}{1+\phi})
\end{align*}
subject to the budget and liquidity constraints:
\begin{align*} 
\text{s.t.} \quad & (1+s_{2t}+\tau_c\frac{CBDC_{2t}}{l_{2t}})c_{2t}+(m_{2Ht}-\frac{1-\delta_m}{\pi_t}m_{2Ht-1})+s_t(m_{2Ft}-m_{2Ft-1})\\
&+(CBDC_{2t}-\frac{r_{t-1}^{CBDC}}{\pi_t}CBDC_{2t-1})  \leq w_th_{2t}+t_{2t}-\frac{\kappa_M}{2}s_t(\lambda m_{2Ft}-\bar{m}_F)^2  \quad (\lambda_{2t})\\
&l_{2t} = ((m_{2Ht})^{\frac{\epsilon_m-1}{\epsilon_m}}+(s_t m_{2Ft})^{\frac{\epsilon_m-1}{\epsilon_m}}+(CBDC_{2t})^{\frac{\epsilon_m-1}{\epsilon_m}})^{\frac{\epsilon_m}{\epsilon_m-1}} \quad (\tau_{2t}\lambda_{2t}) \\
& s_{2t} = z_tA_2\frac{c_{2t}}{l_{2t}}+B_2\frac{l_{2t}}{c_{2t}}-2\sqrt{A_2B_2} \quad (\mu_{2t}\lambda_{2t})
\end{align*}
Taking first order conditions with respect to $c_{2t}, h_{2t}, s_{2t}, l_{2t}, m_{2Ht}, m_{2Ft}, CBDC_{2t}$: 
\begin{align*}
c_{2t}: \quad &c_{2t}^{-\sigma}-\lambda_{2t}(1+s_{2t}+\tau_c\frac{CBDC_{2t}}{l_{2t}})+\tau_{2t}\lambda_{2t}(z_tA_2\frac{1}{l_{2t}}-B_2\frac{l_{2t}}{c_{2t}^2}) = 0, \\
h_{2t}: \quad &-\chi h_{2t}^{\phi}+\lambda_{2t}\omega_t  = 0, \\
s_{2t}: \quad &-\lambda_{2t}c_{2t}-\tau_{2t}\lambda_{2t} = 0, \\
l_{2t}: \quad &\lambda_{2t}\tau_c\frac{CBDC_{2t}}{l_{2t}^2}c_{2t}+\tau_{2t}\lambda_{2t}(-z_tA_2\frac{c_{2t}}{l_{2t}^2}+B_2\frac{1}{c_{2t}})-\mu_{2t}\lambda_{2t} = 0, \\
m_{2Ht}: \quad &-\lambda_{2t}+\beta E_t\lambda_{2t+1}\frac{1-\delta_m}{\pi_{t+1}}+\mu_{2t} \lambda_{2t}(\frac{l_{2t}}{m_{2Ht}})^{\frac{1}{\epsilon_m}}= 0, \\
m_{2Ft}: \quad &-\lambda_{2t}+\beta E_t\lambda_{2t+1}\frac{s_{t+1}}{s_t}+\mu_{2t} \lambda_{2t}(\frac{l_{2t}}{s_tm_{2Ft}})^{\frac{1}{\epsilon_m}}-\lambda_{2t}\kappa_M\lambda(\lambda m_{2Ft}-\bar{m}_F) = 0, \\
CBDC_{2t}: \quad &-\lambda_{2t}(1+\tau_c\frac{c_{2t}}{l_{2t}})+\beta E_t\lambda_{2t+1}\frac{r_{t}^{CBDC}}{\pi_{t+1}}+\mu_{2t} \lambda_{2t}(\frac{l_{2t}}{CBDC_{2t}})^{\frac{1}{\epsilon_m}}= 0.
\end{align*}


\subsection{Bank}
There is a continuum measure of monopolistic competitive banks $j \in [0,1]$. They take deposits from unconstrained households indexed by domestic or foreign currency $d_{Ht}(j)$ and $d_{Ft}(j)$, and invest in intermediate firms $i_t(j)$ and accumulate capital stock $k_t(j)$. The deposit demand functions are solved from the household problem above. 
\begin{align*}
d_{Ht}(j) &= \Biggl(\frac{(r_t-r_t^d(j))^{-\epsilon_b}}{\big(\int_0^1(r_t-r_t^d(j))^{1-\epsilon_b}dj\big)^{-\frac{\epsilon_b}{1-\epsilon_b}}}\Biggl)d_{Ht} \\
d_{Ft}(j) &= \Biggl(\frac{(\frac{r_t^*}{1+\tau_{Bt}}-\frac{r_t^d(j)}{1+\tau_{Dt}})^{-\epsilon_b}}{\big(\int_0^1(\frac{r_t^*}{1+\tau_{Bt}}-\frac{r_t^d(j)}{1+\tau_{Dt}})^{1-\epsilon_b}dj\big)^{-\frac{\epsilon_b}{1-\epsilon_b}}}\Biggl)d_{Ht}
\end{align*}

Banks are owned by unconstrained households, so bankers solve the following optimization problem: 
\begin{align*}
 &\max_{r_t^d(j),i_t(j),k_t(j)}E_0 \sum_0^{\infty}\beta^t\frac{\lambda_{1t}}{\lambda_{10}}(r_t^kk_{t-1}(j)-i_t(j) \\
 &+(d_{Ht}(j)-\frac{r_{Ht-1}^d(j)}{\pi_t}d_{Ht-1}(j))+s_t(d_{Ft}(j)-r_{Ft-1}^d(j)d_{Ft-1}(j)),
 \end{align*}
subject to the law of motion of capital and balance sheet constraint: 
 \begin{align*}
  \quad & k_t(j) = (1-\delta)k_{t-1}(j)+(1-\frac{\kappa_I}{2}(\frac{i_t(j)}{i_{t-1}(j)}-1))i_t(j) \\
& k_t(j) = d_{Ht}(j)+s_td_{Ft}(j).
\end{align*}
The equilibrium is symmetric and the optimal deposit rate chosen by bank is 
\begin{align*}
{r_{Ht}^d}^* &= \frac{\epsilon_b}{\epsilon_b-1}\frac{\beta E_t \lambda_{1t+1}(r_{t+1}^k+(1-\delta)q_{t+1})-(q_t-1)\lambda_{1t}}{\beta E_t (\lambda_{1t+1}/\pi_{t+1})}-\frac{1}{\epsilon_b-1}r_t \\
{r_{Ft}^d}^* &= \frac{\epsilon_b}{\epsilon_b-1}\frac{\beta E_t \lambda_{1t+1}(r_{t+1}^k+(1-\delta)q_{t+1})-(q_t-1)\lambda_{1t}}{\beta E_t (\lambda_{1t+1}s_{t+1}/s_t)}-\frac{1}{\epsilon_b-1}\frac{1+\tau_D}{1+\tau_B}r_t^*
\end{align*}


\clearpage
\section{Calibration}

\begin{table}[h!]
\centering
\begin{tabular}{llll}
\hline\hline
 & Description & Value & Reference \\ \hline
\multicolumn{4}{l}{\bf Household}     \\
$\beta$    & Discount factor       & 0.990  & Gali and Monacelli (2015)              					 \\
$\sigma$     & Intertemporal elasticity of substitution     & 2      \\
$\chi$       & Leisure weight     & 1  & Gali and Monacelli (2015)                                    \\
$\phi$        & Frisch elasticity      & 3     & Gali and Monacelli (2015)                                 \\
$A_1$      & Transaction cost (unconstraint)   & 0              	     \\
$B_1$     & Transaction cost (unconstraint)   & 0                   	 \\
\multicolumn{4}{l}{\bf Firm}     \\
$\alpha$     & Capital share    & 0.33     &   Gali and Monacelli (2015)                       			\\
$\kappa_P$     & Price adjustment cost   & 2                                                      \\
$\epsilon$       & Intermediate-good elasticity of substitution  & 6  &   Gali and Monacelli (2015)       \\
$\delta$       & Capital depreciation rate        & 0.025   & Gertler and Karadi (2011)                                             \\
\multicolumn{4}{l}{\bf Bank}     \\
$\kappa_I$         & Investment adjustment cost        & 1.728       & Gertler and Karadi (2011)                            \\
$\epsilon_b$      & Bank elasticity of substitution    & $\infty$                                     \\
\multicolumn{4}{l}{\bf Open Economy}     \\     
$\eta$      & Domestic and foreign good ES     & 2   &   Gali and Monacelli (2015)                    \\
$\bar{y}^*$  & Foreign output    & 1                                                            \\
$\bar{r}^*$   & Foreign bond interest rate           & $1/\beta$                                   \\
$\rho_{y^*}$  & $y^*$ autocorrelation  & 0.6031   &       George et al. (2020)                                \\
$\rho_{r^*}$   & $r^*$ autocorrelation  & 0.5374     &       George et al. (2020)                                \\
$\sigma_{y^*}$  & Standard deviation of $y^*$ innovation   & 0.0788    &       George et al. (2020)                                \\
$\sigma_{r^*}$  & Standard deviation of $r^*$ innovation & 0.0799    &       George et al. (2020)                                \\\hline
\end{tabular}
\caption{Parameters directly calibrated from literature}
\end{table}

\begin{table}[h!]
\centering
\begin{tabular}{lllll}
\hline\hline
& Description & Value & Implied steady state & Value \\\hline
\multicolumn{5}{l}{\bf Household}     \\
$\lambda$  & Share of constrained households  & 20\%  & Share of unbanked households & 20\%  \\
$\delta_m$ & Cost of carrying cash & 0.2 & Time spent on getting cash \\
$\epsilon_m$ & Cash-CBDC elasticity of substitution &  2 & CBDC adoption rate \\
$A_2$      & Transaction cost (constraint)   & 0.9     & Money velocity    \\
$B_2$     & Transaction cost (constraint)   & 0.7      & Money velocity   & 1.25    \\
\multicolumn{5}{l}{\bf Government}     \\
$\bar{r}$     & Taylor rule: interest rate target            & $\bar{\pi}/\beta$    & Nominal interest rate &      	 \\
$\bar{\pi}$     & Taylor rule: inflation target       & 1.03        & Inflation rate & 3\%                        				 \\
$\overline{\Delta e}$    & Taylor rule: real exchange rate target       & 1.03   & Exchange rate & 1    			  \\
$\tau_C$ & Consumption tax & 0.12 &  \\
$\bar{B}$ & Government bond & 2 & National debt/GDP & 100\% \\
$\bar{G}$ & Government purchase & 0.3 & Government purchase/GDP & 15\% \\
\multicolumn{5}{l}{\bf Open economy} \\
$\gamma$     & Domestic home bias           & 0.58      & Import/GDP ratio             & 	42.13\%                           \\
$\gamma^*$  & Foreign home bias              & 0.27      & Export/GDP ratio             & 27.49\%                                     \\                                                   
$\bar{B}_F$       & Steady state foreign bond level    & 0.05     & Financial account balance/GDP &      0.05              \\
$\kappa_B$     & Foreign bond adjustment cost    & 5       & Financial account volatility &                                       \\
$\bar{D}_f$       & Steady state foreign deposit level   & 0.5  \\
$\kappa_D$      & Foreign deposit adjustment cost     & 3                                         \\
$\bar{M}_f$      & Steady state foreign currency level     & 0.05  &  Dollar/GDP     \\
$\kappa_M$     & Foreign currency adjustment cost     & 3                                         \\      \hline                                    
\end{tabular}
\caption{Parameters with implied steady state implications}
\end{table}

\begin{table}[h!]
\centering
\begin{tabular}{lll}
\hline\hline
& Description & Value \\\hline
\multicolumn{3}{l}{\bf Shocks}     \\                                                              
$\rho_a$      & $a$ autocorrelation    & 0.8552            \\
$\rho_z$       & $z$ autocorrelation     & 0.7217   	    \\
$\rho_g$     & $g$ autocorrelation & 0.8 \\
$\sigma_a$     & Standard deviation of $a$ shock innovation        & 0.0711   \\
$\sigma_z$     & Standard deviation of $z$ innovation   & 0.0694      \\
$\sigma_g$     &  Standard deviation of $g$  innovation & 0.05 \\
$\sigma_m$     & Standard deviation of $m$ innovation   & 0.25              \\
\multicolumn{3}{l}{\bf Taylor Rule}     \\ 
$\rho_r$     & Interest rate elasticity    & 0.5      	 \\
$\phi_{\pi}$    & Inflation elasticity       & 10       	\\
$\phi_y$       & Output elasticity           & 10     	\\
$\phi_e$       & Exchange rate elasticity          & 2      	\\\hline
\end{tabular}
\caption{Estimation}
\end{table}

\clearpage

\section{Welfare analysis}
Table \ref{Welfare1} shows the welfare change if the economy is dollarized. The constrained households can use dollar as an alternative payment method; unconstrained households can choose to index their deposit using dollar. The steady state of economy with and without dollarization is presented in the first and third column of Table \ref{SS}. Dollarization provides additional liquidity service to constrained households. It increases the welfare of constrained households by decreasing the transaction cost and cost of carrying the domestic cash. However, it decreases the welfare of unconstrained households for two reasons:  (1) constrained households decrease labor supply, they have better way to smooth the consumption; (2) real exchange rate is higher, the dead-weight loss caused by carrying cash is lower. Dollarization decreases the overall social welfare, and decreases the tax collected by the government. 
\begin{table}[h!]
\centering
\begin{tabular}{lccc}
\hline\hline
Welfare change & Social  & Unconstrained & Constrained  \\\hline
Dollarization & -0.0015 &  -0.0053 &   0.0082 \\\hline        
\end{tabular}
\caption{Welfare change by adding dollarization}
\label{Welfare1}
\end{table}

Table \ref{Welfare2} shows the welfare change if the CBDC is introduces. CBDC increases welfare of constrained households by decreasing the transaction cost and cost of carrying the domestic cash. The steady state of economy with and without CBDC is presented in Table \ref{SS}. Introduction of CBDC decreases the use of dollar, even though the constrained households can avoid paying tax by using dollar. 

However, CBDC is not Pareto improving, the welfare of unconstrained households decreases when CBDC is introduced for two reasons:   (1) constrained households decrease labor supply, they have better way to smooth the consumption; (2) real exchange rate is higher, the dead-weight loss caused by carrying cash is lower. Unlike dollarization, CBDC increases the overall social welfare, because CBDC does not increase the real exchange rate as much as dollarizaiton. CBDC increases the government revenue in both scenarios, since the government can collect more consumption tax from constraint households. 

The welfare gain of CBDC is more significant when the economy is dollarized. The introduction of CBDC encourages the constraint households to use CBDC instead of dollar, and does not drive up much the real exchange rate. 

\begin{table}[h!]
\centering
\begin{tabular}{lccc}
\hline\hline
Welfare change & Social & Unconstrained & Constrained  \\ \hline
TANK &    0.0020  &  -0.0121 &   0.0400\\
TANK with dollarization &  0.0038  &  -0.0077  &  0.0345 \\\hline
\end{tabular}
\caption{Welfare change by introducing CBDC}
\label{Welfare2}
\end{table}

  
\begin{table}[h!]
\centering
\begin{tabular}{lllll}
\hline\hline
                 &\multicolumn{2}{c}{TANK}  &\multicolumn{2}{c}{Dollarization}  \\
                 &  no CBCD & CBDC & no CBCD &  CBDC  \\\hline
Consumption                 & 1.2924  & 1.2896  & 1.2905  & 1.2901  \\
Labor                       & 0.8766  & 0.8791  & 0.8792  & 0.8798  \\
Capital                     & 15.0967 & 15.0609 & 15.0844 & 15.067  \\
Investment                  & 0.3774  & 0.3765  & 0.3771  & 0.3767  \\
Domestic output             & 2.2424  & 2.2449  & 2.2462  & 2.2464  \\
Consumption (unconstrained) & 1.3735  & 1.359   & 1.3668  & 1.358   \\
Labor (unconstrained)       & 0.8344  & 0.8389  & 0.836   & 0.8391  \\
Deposit                     & 18.8709 & 18.8261 & 10.7379 & 10.6945 \\
Deposit indexed in dollar   & 0       & 0       & 6.25    & 6.25    \\
Consumption (constrained)   & 0.9679  & 1.0121  & 0.985   & 1.0186  \\
Labor (constrained)         & 1.0455  & 1.0401  & 1.0516  & 1.0424  \\
Cash                        & 1.0049  & 0.1697  & 0.3323  & 0.072   \\
Dollar                      & 0       & 0       & 0.1584  & 0.0587  \\
CBDC                  & 0       & 0.4284  & 0       & 0.2774  \\
Transaction cost            & 0.0062  & 0.0001  & 0.0021  & 0       \\
Cost of cash                & 0.1005  & 0.017   & 0.0332  & 0.0072  \\
Return to capital           & 0.0351  & 0.0351  & 0.0351  & 0.0351  \\
Wage                        & 1.2273  & 1.2209  & 1.2228  & 1.2205  \\
Domestic price              & 0.8593  & 0.8563  & 0.8572  & 0.8561  \\
Real exchange rate          & 1.2921  & 1.3016  & 1.2988  & 1.3023  \\
Tax revenue                 & 0.1319  & 0.1396  & 0.1312  & 0.1363 \\\hline
\end{tabular}
\caption{Steady State Comparison}
\label{SS}
\end{table}
Table \ref{SS} compares the steady state of four scenarios. Adoption of CBDC increases the domestic output by promoting financial inclusion. Also, by promoting CBDC, government is able to collect for tax revenue by monitoring better the transaction. By using CBDC, the dead weight loss associated with cash is reduced significantly. 
      
      
\clearpage
\section{Variance Decomposition}
Table \ref{variance} presents the variance decomposition of the benchmark TANK model. The consumption fluctuation of unconstrained households is mainly contributed to TFP shocks, while consumption fluctuation of constrained households is affected more by liquidity and monetary shocks. The terms of trade is heavily affected by the foreign and liquidity shocks.
\begin{table}[h!]
\centering 
\scriptsize
\begin{tabular}{lcccccc}
\hline \hline
                                    & TFP     & Liquidity & Government  & Foreign & Foreign & Monetary \\
                                    &     &  & purchase &  output & interest rate &  \\\hline
Output                      & 96.902  & 0.0633  & 0.2687 & 2.685   & 0.0089 & 0.0721  \\
Consumption                 & 92.9651 & 0.1279  & 0.683  & 1.7427  & 4.067  & 0.4142  \\
Consumption (unconstrained) & 91.4215 & 0.8433  & 0.4131 & 1.3948  & 5.823  & 0.1043  \\
Consumption (constrained)   & 70.2997 & 14.2373 & 0.7096 & 1.4267  & 4.5426 & 8.7842  \\
Interest rate               & 84.5918 & 0.3697  & 0.0102 & 2.0074  & 12.969 & 0.0518  \\
Deposit rate                & 84.5918 & 0.3697  & 0.0102 & 2.0074  & 12.969 & 0.0518  \\
Wage                        & 88.9676 & 0.1126  & 0.0065 & 4.1987  & 0.7264 & 5.9882  \\
Domestic inflation          & 70.2562 & 0.0489  & 0.0647 & 1.671   & 0.0708 & 27.8883 \\
CPI inflation           & 65.1624 & 0.0453  & 0.092  & 1.0422  & 0.4122 & 33.2459 \\
TOT                         & 47.7393 & 1.367   & 0.2858 & 44.7414 & 5.7661 & 0.1004  \\
Deposit                     & 97.6781 & 0.6504  & 0.3239 & 0.6284  & 0.6807 & 0.0385  \\
Cash                        & 80.648  & 0.2562  & 0.4346 & 1.4593  & 6.6019 & 10.6001 \\\hline
\end{tabular}
\caption{Variance decomposition of TANK model}
\label{variance}
\end{table}

\begin{figure}[h!]
\centering
\includegraphics[width=0.8\textwidth]{log_c2}
\caption{Variance decomposition of consumption (constrained)}
\label{log_c2}
\end{figure}

Figure \ref{log_c2} shows the variance decomposition of consumption for the constrained households. Both introduction of CBDC and dollar decreases the effect of TFP on the volatility of the consumption of constrained households. Dollarization increases the exposure of constrained households to the international shocks. CBDC increases the effectiveness of monetary policy. 

\begin{figure}[h!]
\centering
\includegraphics[width=0.8\textwidth]{log_tot}
\caption{Variance decomposition of terms of trade}
\label{log_tot}
\end{figure}

Figure \ref{log_tot} shows the variance decomposition of terms of trade. Dollariation amplifies the effect of international shock on the terms of trade.

\clearpage
\section{Impulse Response}

\begin{figure}[h!]
\includegraphics[width=\textwidth]{TFP}
\caption{Impulse Responses of TFP Shock}
\label{IRF1}
\scriptsize{Notes: Response of key variables to a negative 1 standard deviation TFP shock. Vertical axes indicate percentage deviation from steady state. Horizontal axis indicate quarters after shock. }
\end{figure}
Figure \ref{IRF1} plots the responses of key variable following a negative TFP shock. Introduction of CBDC decreases the responses of output and consumption to TFP. CBDC promotes financial inclusion of constrained household, and decreases the volatility of output and consumption. The constrained households switch more to CBDC following a negative TFP shock, since they are more sensitive to the cost of cash, and rely more on CBDC to smooth out the consumption. The terms of trade responds more to TFP shocks when CBDC is introduced. The higher response of terms of trade is a result of the portfolio adjustment. 

\begin{figure}[h!]
\includegraphics[width=\textwidth]{Monetary}
\caption{Impulse Responses of Monetary Policy Shock}
\label{IRF2}
\scriptsize{Notes: Response of key variables to a positive 1 standard deviation monetary policy shock. Vertical axes indicate percentage deviation from steady state. Horizontal axis indicate quarters after shock. }
\end{figure}
Figure \ref{IRF2} plots the responses of key variable following a positive monetary shock. Introduction of CBDC increases the monetary policy effectiveness. A positive monetary policy shock favors the unconstrained households since they receive higher payoff to their deposit. The introduction of CBDC decreases the inequality effect of monetary policy. 

\begin{figure}[h!]
\includegraphics[width=\textwidth]{Liquidity}
\label{IRF3}
\caption{Impulse Responses of Liquidity Shock}
\scriptsize{Notes: Response of key variables to a positive 1 standard deviation liquidity shock. Vertical axes indicate percentage deviation from steady state. Horizontal axis indicate quarters after shock. }
\end{figure}


\end{document}